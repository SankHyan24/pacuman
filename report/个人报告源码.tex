\documentclass[utf8]{ctexart}
\usepackage{graphicx}
\title{个人总结}
\author{孙川}
\date{\today}
\begin{document}
\maketitle
\tableofcontents
\section{主要工作}
在工作中,因为是我提出要制作吃豆人游戏,于是我负责了游戏的设计,代码规范的制定,以及游戏的项目分工。\par
在实际工作中,由于有着比较多的经验,我承担了大程设计中绝大部分的代码任务(动画组件pmanima.c,排行榜pmrank.c,游戏引擎pmrun.c,最短路径算法pmsp.c,地图初始化pmminit.c,方向判别pmdirrt.c,地图编辑器pmeditor.c,主界面pmLauncher.c,游戏画面pmgame.c)。负责设计完成了除帮助界面以外的一切图形界面、游戏运行机制、AI引擎、排行榜、地图编辑器的制作等等工作。
\section{经验和教训}
在本次大程序设计中,我的主要工作是负责游戏的设定,以及游戏的后端运行模块的代码。但是在程序设计的过程中,我发现在没有前端同学提供好图形界面之前,后端的一切工作都无法完成测试。这是项目设计过程中的经验教训:由于这个原因,我提前把前端动画组件做好了,导致了曲景邦同学没有工作做,加之另外一个同学很难联系上,所以我的工作量其实增大了很多。\par
在调试的过程中,程序出现了秒退的bug。但是由于程序体量很大,很难再发现是哪里出现的问题。这导致寻找问题出现的原因这件事花费了我们很多精力。这件事情至少让我明白了要有前瞻性,在程序体量还很小的时候,尽可能预先想到要测试的点,避免在后来出现不可预测性错误的时候难以找出问题的实际原因。\par
在完成基本的程序框架以及自研的程序接口之后,我将游戏运行的任务交给了曲景邦同学。但是由于我们交流并不充分,我觉得我的合作者已经掌握了调用封装好的程序的方法,并能够自己写出相应的函数接口并调用,但实际上并没有。相对于之前的代码,修改后的代码变得不规范,而且由于不会合理调用接口,导致程序冲突。这件事情告诉我,制定一份代码规范文件,要比盲目进行合作开发更加重要\par
\section{特色}
因为对游戏设计很感兴趣,就像在大程大作业中制作一个游戏,挑战一下。加上自学了迪杰斯特拉算法,想要在实际项目中试验一下。于是我在征求了老师的意见后选择了这一课题。\par
\subsection{界面设计}
在游戏的设计过程中,我没有一味地选择对原版“吃豆人游戏”全盘照搬,而是在其基础上进行了创新与完善:\par
在游戏的预先设计中,我借鉴了一款发布于2003年的优秀的游戏“Warcraft3(魔兽争霸3)”的页面设计——自定义选择地图,选择人物,地图的预览,以及地图编辑器功能。\par
\includegraphics[scale=0.4]{ss1.png}
这些功能让游戏玩家在游玩的过程中有了更多的自由性和可玩性,大大改善了以往吃豆人被动闯关式的游戏体验。\par
\subsection{AI设计}
在游戏史上,吃豆人在众多方面都具有开创性,其中游戏AI是其中最为重要的部分之一。原版吃豆人四个鬼魂具有四种不同的行为逻辑,但是在自定义地图的情况下,部分AI的功能很难进行实现(比如说在玩家前方四格处埋伏)。\par
我设计了2种简单的行为逻辑——“追逐玩家”和“随机行走”,并基于这两种逻辑发展出了四种不同的行动逻辑——“大多数时候一直追逐”“偶尔追逐”“一直追逐,但在距离玩家足够近时选择随机行走”“随机行走,但在玩家距离足够近的情况下一直追逐玩家”。这四种不同的行为逻辑赋予了游戏NPC不同的性格,带给玩家不同的游戏体验。
\subsection{游戏文化和平衡性}
在游戏的设计末期,我增加了选择游戏人物的功能选项(这在原版吃豆人中是不存在的)。除了原版的黄色的普通吃豆人,新增加的游戏人物“布鲁斯班纳博士”和“尼奥”,一个来自漫威电影的超级英雄“绿巨人”,一个是具有赛博朋克风格的电影“黑客帝国”中的主角“救世主”。这两个人物都是大家感兴趣的。将这两个人物融合进游戏中,让游戏有了“故事
和内涵,体现了游戏文化。\par
然而,游戏人物的加入显然导致了游戏平衡性失调这一问题。对此,我们通过改变吃豆人的吃豆目标,将游戏的平衡性调整到一个合理的区间范围内。\par
\subsection{“彩蛋”设计}
一个有意思的游戏往往有着一些隐藏的功能,需要玩家自己发现,也就是所谓的“彩蛋”。\par
在新版的devc环境下测试本程序,玩完一局之后重新开局一般不会发生秒退的问题。因此玩家可以选择布鲁斯班纳博士,在游戏的过程中变身,然后点击暂停退出游戏,此时的玩家获得隐藏人物“永久版绿巨人”。再次进入游戏时,将一直保持绿巨人形态,直到推出游戏或更改人物。
\section{最终总结}
在本次c程序设计中,我第一次体验了大型程序设计以及团队合作的程序设计,收获颇多。\par
玩游戏给我带来很多的乐趣,但是显然,设计游戏给我带来的乐趣更大一些。我想,这个大作业应该已经体现了我对此图形库的运用能力。开发的过程中,一直想让我们的程序作业更加完美,但是最终还是有着一些小小的瑕疵。但总之,我举得本次大程设计很成功,我很满意。\par
\end{document}